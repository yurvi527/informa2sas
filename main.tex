\documentclass{article}
\usepackage[utf8]{inputenc}
\usepackage[spanish]{babel}
\usepackage{listings}
\usepackage{graphicx}
\graphicspath{ {images/} }
\usepackage{cite}

\begin{document}

\begin{titlepage}
    \begin{center}
        \vspace*{1cm}
            
        \Huge
        \textbf{Informe Parcial I}
            
        \vspace{0.5cm}
        \LARGE
        Subtítulo
            
        \vspace{1.5cm}
            
        \textbf{Fabian Hoyos Zabala}
        
        \vspace{1.5cm}
        
        \textbf{Karen López}
        
        \vspace{1.5cm}
        
        \textbf{Yuribia Arroyave}
            
        \vfill
            
        \vspace{0.8cm}
            
        \Large
        Departamento de Ingeniería Electrónica y Telecomunicaciones\\
        Universidad de Antioquia\\
        Medellín\\
       Abril de 2021
            
    \end{center}
\end{titlepage}

\tableofcontents
\newpage
\section{Sección Introductoria}\label{intro}

\section{Contenido} \label{contenido}

\subsection{Análisis del Problema}

\subsection{Esquema para el desarrollo del algoritmo}


\subsection{Algoritmo inplementado}

\subsection{Problemas que se presentaron}

\subsection{Evolución y consideraciones}

\section{Inclusión de imágenes} \label{imagenes}

En la Figura (\ref{fig:cpplogo}), se presenta el logo de C++ contenido en la carpeta images.

\begin{figure}[h]
\includegraphics[width=4cm]{cpplogo.png}
\centering
\caption{Logo de C++}
\label{fig:cpplogo}
\end{figure}

Las secciones (\ref{intro}), (\ref{contenido}) y (\ref{imagenes}) dependen del estilo del documento.

\bibliographystyle{IEEEtran}
\bibliography{references}

\end{document}
