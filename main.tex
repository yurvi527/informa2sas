\documentclass{article}
\usepackage[utf8]{inputenc}
\usepackage[spanish]{babel}
\usepackage{listings}
\usepackage{graphicx}
\graphicspath{ {images/} }
\usepackage{cite}

\begin{document}

\begin{titlepage}
    \begin{center}
        \vspace*{1cm}
            
        \Huge
        \textbf{Informe Parcial I}
            
        \vspace{0.5cm}
        \LARGE
        Subtítulo
            
        \vspace{1.5cm}
            
        \textbf{Fabian Hoyos}
        
        \vspace{1.5cm}
        
        \textbf{Karen López}
        
        \vspace{1.5cm}
        
        \textbf{Yuribia Arroyave}
            
        \vfill
            
        \vspace{0.8cm}
            
        \Large
        Departamento de Ingeniería Electrónica y Telecomunicaciones\\
        Universidad de Antioquia\\
        Medellín\\
       Abril de 2021
            
    \end{center}
\end{titlepage}

\tableofcontents
\newpage
\section{Sección Introductoria}\label{intro}

\section{Contenido} \label{contenido}

\subsection{Análisis del Problema}

Se necesita crear una aplicación para un puesto en una empresa llamada Informa 2 S.A.S, donde se requiere desarrollar una animación, la empresa presenta una dificultad y es que tiene un limitante de arduinos, los que poseen no cunetan con suficientes puertos digitales, se debe lograr una animación que muestre letras o una figura que el usuario ingrese. Dicha animación debe tener un patrón de leds. (64leds), se debe conectar y controlarlo con un arduino y el integrado 74HC595, la matriz de leds debe de ser de 8x8, se deben mostrar esos patrones y lograr realizar la conexión del sistema operativo con los 64 leds siguiendo la estructura dada. Para lograr esto se debe realizar una función que pida un patrón por la consola serial y mostrarlo. El objetivo principal es crear una función que muestre cuantos patrones se quieren mostrar y pida los patrones.



\subsection{Esquema para el desarrollo del algoritmo}


\subsection{Algoritmo implementado}
// 74HC595 = Pines de Arduino
const int dato = 2;
const int reloj = 3;
const int paso = 4;

//Prototipo de funciones
void verificacion(int); //Encender todos los leds
void imagen(int);//Mostrar un patron
/*void publik ();//Publicar secuencia de patrones 

ingresados por usuario
void bindec(int); //Convertir binario a decimal*/
            

// Contador de columnas
int j = 0;

// Contador de duración de secuencia
int k;
int fila[8] = {127, 191, 223, 239, 247, 251, 253, 

254};


// columnas de prueba que ayudaron a verificar 

funcionamiento de leds y codigo
int columnaV[8] = {255, 255, 255, 255, 255, 255, 

255, 255}; //encendido completo

int columnaF[8] = {60, 66, 165, 129, 165, 153, 

66, 60}; // Emoticon Feliz

int columnaN[8] = {60, 66, 165, 129, 129, 189, 

66, 60}; // Emoticon Normal

int columnaT[8] = {60, 66, 165, 129, 153, 165, 

66, 60}; // Emoticon Triste

// Se inicializa el 74HC595, y los puertos 

digitales del arduino
void setup()
{
  Serial.begin(9600);
 
  pinMode(dato, OUTPUT); // dato
  pinMode(reloj, OUTPUT); // reloj
  pinMode(paso, OUTPUT); // paso
  
  pinMode(13, OUTPUT);
}


void loop()
{
   verificacion(columnaV); //Encendido total con 

la función verificación
  //imagen(columnaV);//Encendido total con 

funcion imagen
   //imagen(columnaN); //emoticon cara normal
   //imagen(columnaT); //emoticon cara triste
  //imagen(columnaF); //emoticon cara feliz
  //Serial.println(binary, analogValue);
   
}
  
// Funcion para imprimir un patron, definido por 

el vector columna(N,T,F y lo que ingrese el 

usuario)
//Se usa el nombre del vector de columnas, 

como puntero de la entrada cero del mismo 

vector
void imagen(int tipo[8])
{
  for(k = 0; k<100; k++)
  {
    for(int i=0; i<8; i++)
    {
      digitalWrite(reloj, LOW);//Se baja con un 

pulso el reloj
      shiftOut(dato, paso, MSBFIRST, *(tipo

+j));//Trae lo que hay en el vector columnas, por 

medio del contenido de su puntero
      shiftOut(dato, paso, MSBFIRST, fila[i]);//La 

fila trae tiempos de duración de la secuencia
      digitalWrite(reloj, HIGH);
      j++;
    delay(0.1);  
    }
    j = 0;
    
  }
  
}
//Se usa el nombre del vector de columnas, 

como puntero de la entrada cero del mismo 

vector
void verificacion(int columnaV[8])
{
  for(int k = 0; k<100; k++)
  {
    for(int i=0; i<8; i++)
    {
      digitalWrite(reloj, LOW);//Se baja con un 

pulso el reloj
      shiftOut(dato, paso, MSBFIRST, *(columnaV

+j));//Trae lo que hay en el vector columnas, por 

medio del contenido de su puntero
      shiftOut(dato, paso, MSBFIRST, fila[i]);//La 

fila trae tiempos de duración de la secuencia
      digitalWrite(reloj, HIGH);
      j++;
    delay(0.1);  
    }
    j = 0;
    
  }
  
}
//conversor de binario a decimal
/*void bindec(int binario){
    int result, resto=0;
    int digito[8];
    //cout <<"Ingrese binario: ";
    //cin>> binario;
    for(int i=0;i<8;i++){
        digito[i] =binario %10;
        binario /=10;
    }
    for (int i=7; i >=0; i--){
        result = (resto *2)+digito[i];
        resto = result;
     }
        //cout<<result;
 return(result);
} */           

/*//Funcion para mostrar una secuencia de 

patrones
void publik (){
  unsigned int cant;
  Serial.print("Esriba la cantidad de patrones que 

desea ingresar: ");
  Serial.println(cant);
  int patrones[8][cant];
  for (int i=1; i<=cant;i++)
  {
    Serial.print("Ingrese el patron", i, ". ");
    //cargara una fila de la matriz patrones,con la 

información de cada columna
  }
}*/


\subsection{Problemas que se presentaron}

\subsection{Evolución y consideraciones}

\section{Inclusión de imágenes} \label{imagenes}

En la Figura (\ref{fig:cpplogo}), se presenta el logo de C++ contenido en la carpeta images.

\begin{figure}[h]
\includegraphics[width=4cm]{cpplogo.png}
\centering
\caption{Logo de C++}
\label{fig:cpplogo}
\end{figure}

Las secciones (\ref{intro}), (\ref{contenido}) y (\ref{imagenes}) dependen del estilo del documento.

\bibliographystyle{IEEEtran}
\bibliography{references}

\end{document}
